El agua rota

I. Una inspiración, una espiración lejana,
una oración, un mantra.

II. El agua está rota y se esparce
en la sonrisa de los incrédulos,
en los hombros de los que penan,
en las cabezas de los temerosos.

No me arrodillo,
accedo al pulpito y reparto el remedio,
renuncio al lujo y al bienestar
y me coloco en el centro,
donde todos me vean,
donde él 
pueda llegar a mi sin esfuerzo.

Mi huella apenas alcanza a contemplar
un segundo de la vida que pensaba otorgarle.
Mi espalda se curva ante el peso de la historia
que puedo leer en mis sueños.

No soy la fuente,
soy un canal de la fuente.
No soy la luz,
soy un faro.
No permito que la voz se levante en mi interior,
pues debe repartirse por todo lo que soy y lo que somos.

No soy el que recibe la bendición,
soy el que bendigo.

El que ha de reinar,
me ha entregado el agua rota de la bendición no acordada,
el que articula el tiempo y escribe en las entrañas,
se muestra impasible ante la superstición de los extraños.

Sabed, aquellos que teméis lo que ha de venir,
que no hay necesidad de construir grandes templos
ni enormes catedrales,
la grandeza se acumula en el extremo del cabello más pequeño,
de un animal que vive en un futuro que no alcanzaremos.

III. Los dragones hambrientos
devoran las flores que trajiste
para celebrar que hemos conseguido nacer de nuevo,
en un nuevo mundo,
en un infinito e incesante tintineo de augurios
que consideramos como lo más profundo.

Distraídos, enlazamos nuestras manos al unísono
para festejar la quema de todo aquello
por lo que creíamos que luchábamos.

Primero los ojos se apagan,
luego la voz.
Las manos pierden 
su sensible capacidad de percibir lo bueno,
el gusto cae luego,
el olfato continua intacto,
cae el último en el pozo negro
del que nadie podrá ya salvarnos.

IV. Fue entonces cuando alguien recordó, 
que había dejado marcado el camino
con el agua que alimentó al primero de los seres vivos.
El agua que no se borra de la faz de la tierra
ni de ningún otro material de la existencia.
El agua que una vez creó 
e imaginó todo lo que existe
te fue entregada en el futuro
para que volvieses a por nosotros.

Aquello que esperábamos
ha venido a buscarnos y,
aún estando preparados,
no queremos acompañarle.

El color es un invento,
la forma perece y cambia,
pero lo que no se ve permanece,
lo que se siente con el ojo
de la férrea convicción permanece.

No pude saciar mi sed
con el agua rota que me fue entregada.
Quizá por ignorancia,   
quizá por miedo,
quizá por creerme más listo
o más fuerte
que todos ellos.

V. Los hombros llenos de dolor
y las cabezas borrachas de información
y pensamientos,
amontonan delirios de purificación.
Las rodillas sienten el placer
de no tener que dirigir su camino
hacia el destino que para ellas escribieron los antiguos.

Las espinas quietas en la alfombra
no logran retener el movimiento,
así como todo el poder de los hombres
y los muros de cemento
no impiden el vuelo de un pequeño pájaro,
que escapa en el último momento.

VI. De alguna forma,
agradecí más a aquel que me dejó sediento
que al que me dio de beber.
De alguna forma, 
contemplé en la espalda los secretos
que no lograba ver en el pecho.
De alguna forma,
comprendí que lo malo sostiene a lo bueno,
y la luz en lo alto no brilla
si algo no se apaga luego.

