El caviar de los enfermos 

Busco un sitio para nacer de nuevo
y despertar de este sueño.

Dos millones de años después y no sabemos nada,
solo creemos que sabemos algo.
Comemos el caviar de los enfermos,
y nos arrodillamos antes de tiempo.

No importa la verdad,
a nadie le importa la verdad,
todos fingimos.
Vivimos en una enorme granja
de titanio deslumbrante,
de sensaciones lejanas que creemos intensas.

Demasiado esfuerzo,
demasiado humanos.
Buscamos a alguien que reme con nosotros hacia el mas allá,
pero estamos solos.

No hay un dios,
no hay reencarnación,
no hay recuerdos,
no hay una vida tangible,
las almas se escapan del cuerpo pues
no pueden, 
ni deben,
soportar el sufrimiento ajeno.

Busco un sitio para nacer de nuevo
y despertar de este sueño.

El loco es libre,
el inadaptado es libre,
el esclavo es libre en su pensamiento
y anhela ser azotado.

Infeliz del destino,
incrédulo del azar que todo lo rige,
embriagado por las mentiras de ruiseñor,
viperinas a tiempo completo,
de un sistema mil veces denunciado.
El más sabio en lo alto del púlpito habla,
y nadie escucha,
y nadie cree.

Busco un sitio para nacer de nuevo
y despertar de este sueño.


Sin comprender las razones de la cueva de las sombras,
creces en la espesura de lo inaudible,
contemplas las heridas y los deseos,
placer, provocación, regalos que no se abren,
o que se abren demasiado rápido,
rabietas, desengaños, agujeros hasta lo más profundo.

Pero hay esperanza, pensabas,
buscando en las miradas,
avergonzado por no haber escuchado antes.
¿Que podías hacer?

Tocarte el ombligo,
tocarte el ombligo y despertar.